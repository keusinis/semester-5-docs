\documentclass{article}
\usepackage{graphicx} % Required for inserting images

\title{Vilnius University\\Faculty of Mathematics and Informatics\\Software Engineering\\3rd course\\[1cm] \Huge Universal POS application for small to medium-sized businesses\\[2cm]}
\author{TEAM:\\[0.25cm]Rytis Tauras\\Matas Čeplinkas\\Emil Duko\\Žygimantas Vidmantas\\Rytis Jonika\\[3cm]}
\date{September 2025}

\begin{document}

\maketitle

\newpage
\setcounter{tocdepth}{2}
\tableofcontents
\newpage

\section{Introduction}
\subsection{Software title}
\subsubsection{Full title} Universal POS application for small to medium-sized businesses
\subsubsection{Short title} POS application
\subsection{Problem statement} Today, the point-of-sale (POS) market is faster and busier than ever before. Due to fast pace and high demand, PoS companies and their employees face the challenge of unreliable work systems. To be more precise, the PoS operators who work hand in hand with the system often struggle with unexpected errors, hard to navigate user interfaces, slow processing and inefficient workflow all of which gives rise to unwanted stress on daily basis. Furthermore, it also has a negative effect on business owners, because starting a business takes more time and usually they have to pick through different systems to find the best fit, which results in that process consuming a significant part of funding for installation and employee training. What is more, the clients of these businesses greatly depend on the quality of service and these faulty systems directly contribute to longer wait times and overall diminishment of provided service quality. These challenges mostly arise in cafes, bars, restaurants, spas and other settings where smooth workflow between sales, reservations and management is crucial. The root of the problem usually lies in poor synchronization between different sectors, limited language and problem support as well as confusing or unintuitive interfaces causing the delays and mistakes. It is important to address this problem because faulty POS systems hinder business processes by decreasing profit resulting in financial losses as well as reducing customer satisfaction causing them to have mistrust, which builds barriers for new businesses or enhances frustration for more established business owners and their employees.
\newpage

\section{User needs analysis}
\subsection{Expectations of the stakeholders}
\subsubsection{Primary stakeholders}
\textbf{POS operators} expect to have a system for fulfilling orders and making appointments that:
\begin{itemize}
    \item Has high transaction efficiency
    \item Reduces errors
    \item Helps provide quick assistance to customers
    \item Has minimal training and is easy to remember
    \item Has smooth integration of appointment system
    \item Allows resource monitoring
    \item Supports card, cash and gift-card payments
    \item Supports multiple languages.
\end{itemize}


\subsubsection{Secondary stakeholders}

\textbf{Clients} expect the order/appointments to be accurately delivered and without delays as well as view the ones that are assigned to them. \\ 
\textbf{\\ Other employees} expect the system to be reliable, so as not to interrupt their own work.

\textbf{\\ Business managers} expect the system to show accurate data/statistics about business processes.

\subsubsection{Tertiary stakeholders}

\textbf{Business owners} expect a system that serves customers faster, is error-free, is adaptable to their business model, can be integrated onto current systems and into current workflows seamlessly, and does not require lengthy training sessions for PoS operators.

\subsubsection{Competitors}The closest competitors available to PoS application are:
\begin{itemize}
    \item Paysera POS
    \item winPOS
    \item iiko
    \item r\_keeper
    \item N\_soft
\end{itemize}
The main advantages of our PoS application are multi-language (beyond Lithuanian and English) support, easier to learn and use user interface and more affordable service.
\subsubsection{Developers \& maintenance team}
Team responsible for maintenance and development expects a system that can be scaled or adjusted if business model of PoS system's customer changes. Additionally, it should be maintainable without dedicating more than one person to keep things running.  
\end{document}
